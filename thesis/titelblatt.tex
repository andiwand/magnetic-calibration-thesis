\newgeometry{bottom=2cm, top=2cm, left=3cm, right=2cm}
\begin{titlepage}


\begin{tabular}{ >{\centering}p{9cm} >{\centering}p{7cm} }
	\space & {\line(1,0){120}\\Unterschrift Betreuer}
\end{tabular}

\begin{center}

% Upper part of the page
\begin{figure}[h]
	\centering
	\includegraphics[width=0.5\textwidth]{figures/TU_Wien_Logo.pdf}
	 %Logo gracefully taken from http://www.tuwien.ac.at/dle/pr/publishing_web_print/corporate_design/tu_logo/
\end{figure}

\vspace{\stretch{1}}
\begin{LARGE}

\par\noindent
DIPLOMARBEIT

\vspace{\stretch{1.6}}

\textbf{Particle Filter für automatische Laufzeit-Kalibrierung des 3-Achsen-Magnetometers in Smartphones} \\

\end{LARGE}

\vspace{\stretch{1.6}}
\begin{large}
ausgeführt am Institut für Angewandte Physik (IAP) \\
der Technischen Universität Wien

\vspace{\stretch{0.5}}

unter der Anleitung von \\
\textbf{Ao.Univ.Prof. Dipl.-Ing. Dr.techn. Martin Gröschl}

\vspace{\stretch{0.5}}

und Mitbetreuung von \\
\textbf{Dipl.-Ing. Dr.-Ing. Paolo Fogliaroni} \\
Esri R\&D Center Vienna

\vspace{\stretch{1}}

durch \\

\vspace{\stretch{0.3}}

\textbf{Andreas Stefl, BSc} \\

\vspace{\stretch{2}}

\begin{tabular}{ >{\centering}p{7cm} >{\centering}p{7cm} }
\centering
Wien, \today & \line(1,0){120}\\Unterschrift Student
\end{tabular}
\end{large}

\end{center}
\end{titlepage}
\restoregeometry
