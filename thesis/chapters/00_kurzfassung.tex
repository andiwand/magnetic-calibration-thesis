Heutige Smartphones kommen im Normalfall mit einem eingebauten Magnetometer, welches dazu benutzt werden kann, die horizontale Orientierung, wie mit einem Kompass, zu bestimmen. Das Smartphone besteht jedoch selbst aus magnetischen Materialien, die die Messungen des Sensors systematisch beeinflussen. Eine Kalibrierung ist notwendig, um die Abweichung zu bestimmen und von der Messung abzuziehen. Diese Kalibrierung wird üblicherweise durch die Kombination mehrerer Sensoren, wie Beschleunigungssensor und Gyroskop, erreicht und vom Betriebssystem oder Hersteller durchgeführt. Außerdem ist sie dafür optimiert, wenig Prozessorleistung in Anspruch zu nehmen und wird durch schnelle Rotationen des Smartphones in verschiedene Richtungen, welche durch den Benutzer durchzuführen sind, erreicht. Da es sehr viele verschiedene Geräte auf dem Markt gibt, weichen die Implementierungen voneinander ab und es ist nicht klar wann die Kalibrierung fertiggestellt ist.

In dieser Arbeit stellen wir einen Particle Filter für die kontinuierliche Kalibrierung des Magnetometers vor, der keine besonderen Handbewegungen des Benutzers voraussetzt. Da unser Filter mehrere Möglichkeiten der Kalibrierung gleichzeitig in Betracht zieht, erwarten wir eine sinnvolle Abschätzung des Fortschritts.

Wir werden zeigen, dass unsere Kalibrierung des Magnetometers in Szenarien wie Navigation, Lokalisierung und Wegfindung für Fußgänger jener des Betriebssystems weitgehend überlegen ist. Gleichzeitig wird eine große Bandbreite an Geräten unterstützt.
