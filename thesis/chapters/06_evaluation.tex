In this chapter, we evaluate our implementation of the hard iron calibration presented in Chapter \ref{ch:impl}. First, we give an overview of the hardware used for evaluation. Then we take a look at the behaviour of the sensor data collection and a qualitative evaluation of the orientation filter. Finally, we will evaluate the quality of our hard iron calibration in comparison to the operating system, the error estimation, and its performance.

\section{Hardware}

Four different devices were available for evaluation. They are listed in Table \ref{tbl:hardware}. It was important to compare the results between multiple devices since sensors and hard iron calibration algorithms can vary greatly between them. Apart from that, we need be sure that the performance of our implementation is good enough for real-time processing across different devices to get maximal compatibility.

All these devices run a fairly recent Android version and have been released during the previous 5 years.

\begin{table}[h]
    \centering
    \begin{tabular}{ | l | r | r | }
    \hline
    \textbf{Name}     & \textbf{Release year} & \textbf{Android version} \\ \hline
    Google Pixel 3    & 2018 & 11 \\ \hline
    Google Pixel 2    & 2017 & 11 \\ \hline
    Samsung Galaxy S7 & 2016 & 8 \\ \hline
    LG Nexus 5X       & 2015 & 8.1 \\ \hline
    \end{tabular}
    \caption{Devices used for testing and evaluation.}
    \label{tbl:hardware}
\end{table}

Another possible test device was the Nokia 3 released in 2017 but unfortunately it does not provide uncalibrated magnetic field sensor readings. Up to this point we are not sure about the market share of devices that support uncalibrated readings. Alternatives will be discussed in Chapter \ref{ch:outlook}.

\section{Data collection}
\label{sec:eval_sensor}

The validation of the sensor data collection was important since all the following steps depend on it. Apart from that it was beneficial for building up intuition for the data which is used for filtering.

As a first step only the timestamps of the magnetometer readings were used to calculate the measurement interval. Since Android is not a real-time system the interval will not be constant. Statistics of the interval are given in Table \ref{tbl:mag_interval}.

\begin{figure}[hbt!]
    \centering
    \includegraphics[width=1.0\textwidth]{figures/interval_nexus5.png}
    \caption{Histogram of the of the sampling interval for the magnetometer in the LG Nexus 5X.}
    \label{fig:interval}
\end{figure}

\begin{table}[h]
    \centering
    \begin{tabular}{ | l | r | r | r | r | r | r | r | }
    \hline
    \textbf{Device}   & \textbf{mean} & \textbf{std} & \textbf{min} & $\bm{Q_1}$ & \textbf{median} & $\bm{Q_3}$ & \textbf{max} \\ \hline
    Google Pixel 3    & 10.0 & 3.2 & 0.1 &  9.0 & 10.0 & 11.2 &  57.4 \\ \hline
    Google Pixel 2    &  9.6 & 1.6 & 0.0 &  9.2 &  9.6 &  9.9 &  63.9 \\ \hline
    Samsung Galaxy S7 & 10.0 & 2.6 & 0.3 &  9.3 & 10.0 & 10.7 & 115.5 \\ \hline
    LG Nexus 5X       & 19.7 & 2.6 & 0.8 & 19.0 & 19.7 & 20.4 &  86.5 \\ \hline
    \end{tabular}
    \caption{Statistics of the magnetometer reading interval in $ms$.}
    \label{tbl:mag_interval}
\end{table}

Secondary, the readings of the magnetometer were displayed as a three-dimensional scatter plot to reveal the hard iron effect. For this purpose the device has been rotated in multiple directions while reading the magnetic field. Two-dimensional projections of such a three-dimensional scatter are shown in Figure \ref{fig:scatter} for the Samsung Galaxy S7.

\begin{figure}[hbt!]
    \centering
    \includegraphics[width=1.0\textwidth]{figures/scatter_s7.png}
    \caption{Scatter plot of the magnetometer readings while rotating the Samsung Galaxy S7 in multiple directions.}
    \label{fig:scatter}
\end{figure}

Without the hard iron effect we would expect the center of the plots at the origin. The maximal distance from the origin should be the magnitude of the Earth's magnetic field if other environmental sources are neglectable. Without considering the hard iron effect the estimation of the horizontal orientation would be meaningless.

\section{Orientation filter}

Since the hard iron calibration depends highly on the orientation estimation it was necessary to validate the implementation and to evaluate its performance. A quantitative evaluation would require reference values for the orientation of the device for comparison. That could be achieved by tracking visual landmarks with a camera, rotating the device with a robotic arm or simulation of sensor data. Such experiments were carried out at the Department of Geodesy and Geoinformation at TU Wien.\cite{Ettlinger2018}

The implementation of the orientation filter needed at least a qualitative evaluation to guarantee its functionality. That was carried out with an augmented reality demo application. A cube was placed in the center of a 3D scene and the position and orientation of the camera in this scene was calculated based on the orientation of the phone in such a way that the cube seems to be stationary in world coordinates. This approach validates the local and global frame of reference, the transformation between them and the response to sensor updates.

The 3D scene is illustrated in Figure \ref{fig:cube_scene}. Pictures of the augmented cube rendered by the LG Nexus 5X after rotation are show in Figure \ref{fig:cubes}.

\begin{figure}[hbt!]
    \centering
    \includegraphics[width=0.6\textwidth]{figures/cube_scene.png}
    \caption{Illustration of the cube and the camera in the 3D scene.}
    \label{fig:cube_scene}
\end{figure}

\begin{figure}[hbt!]
    \centering
    \begin{subfigure}{0.3\textwidth}
        \includegraphics[height=1.5\linewidth]{figures/cube.jpg}
        %\caption{image1}
        %\label{fig:1}
    \end{subfigure}
    \begin{subfigure}{0.3\textwidth}
        \includegraphics[height=1.5\linewidth]{figures/cube_right.jpg}
        %\caption{image1}
        %\label{fig:1}
    \end{subfigure}
    \begin{subfigure}{0.3\textwidth}
        \includegraphics[height=1.5\linewidth]{figures/cube_bottom.jpg}
        %\caption{image1}
        %\label{fig:1}
    \end{subfigure}
    \caption{Pictures of the augmented cube after rotating the device.}
    \label{fig:cubes}
\end{figure}

Since the interval of the sensor data is sampled to 50 ms, little delay and low refresh rate is expected. Apart from that the experience was very positive.

The orientation filter has only a single parameter \textsc{beta} apart from the update interval which models the reliability proportion between accelerometer and gyroscope measurements. This parameter has to be chosen carefully and in the best case individual for each phone or sensor composition since the expected error highly depends on that.

The test application can be used to hand-tune the parameter for a given device.

\section{Hard iron calibration}
% visualize particle cloud over time

% left over oscillations might be due to latency in orientation estimation or soft iron effects (read somewhere that magnetization only takes ns so that might be bs)
%  - rather small ~1µT so not that important for orientation but never the less interesting

In order to evaluate our hard iron calibration, a test track with labels at each turn was set up as a spatial reference for sensor data collection across different devices. The recorded data was later used to simulate the sensor input and evaluate the output of our filter. The track had a total length of approximately 24 m with 4 turns and a total horizontal rotation angle of approximately 540$^{\circ}$. It is illustrated in Figure \ref{fig:eval_scenario}. The \textsc{tick} button in the test application (see Figure \ref{fig:app}) was used to mark the passing of each label.

\begin{figure}[H]
    \centering
    \includegraphics[width=1.0\textwidth]{figures/scenario.png}
    \caption{Illustration of the test scenario.}
    \label{fig:eval_scenario}
\end{figure}

At the beginning the device was placed on a table and a magnet was used to influence the hard iron effect. Afterwards the track was started with the first hit of the \textsc{tick} button with an average walking speed of approximately 1 m/s. At the end of the track the button was pressed one last time and the devices was rotated in multiple directions on the same position to achieve an optimal calibration for reference.

The different phases of the recording are distinguishable because the \textsc{tick} button was pressed in between and the timestamp of that event was stored along the sensor data.

Table \ref{tbl:eval_params} contains the parameters used for the following evaluation.

\begin{table}[H]
    \centering
    \begin{tabular}{ | l | r | }
    \hline
    \textbf{Parameter}              & \textbf{Value} \\ \hline
    \textsc{seed}                   & random \\ \hline
    \textsc{population}             & $10^5$ \\ \hline
    \textsc{deltaTime}              & 50 ms \\ \hline
    \textsc{initialVariance}        & $(100\ \mu T)^2$ \\ \hline
    \textsc{driftRate}              & $1.0$ \\ \hline
    \textsc{predictionVariance}     & $(5\ \mu T)^2$ \\ \hline
    \textsc{minimalRotation}        & $0.1$ \\ \hline
    \textsc{resamplingRate}         & $0.01$ \\ \hline
    \end{tabular}
    \caption{Chosen parameter for evaluation.}
    \label{tbl:eval_params}
\end{table}

Least squares was used to fit a sphere\cite{Jekel2016} to the uncalibrated magnetometer readings for the last part of the recording. The least squares estimate was used for comparison with the estimates of our particle filter.

The following Figures \ref{fig:eval_simulation_pixel3}, \ref{fig:eval_simulation_pixel2}, \ref{fig:eval_simulation_s7}, and \ref{fig:eval_simulation_nexus5x} show the \textcolor{blue}{estimated hard iron bias and error} over time compared to the \textcolor{red}{least squares} estimate. The vertical lines represent the time when a label on the test track was passed.

The following Tables \ref{tbl:eval_simulation_pixel3}, \ref{tbl:eval_simulation_pixel2}, \ref{tbl:eval_simulation_s7}, and \ref{tbl:eval_simulation_nexus5x} contain the values of the estimated hard iron bias and error after passing the labels on the test track from different methods. \textsc{PF} is our real-time particle filter estimate, \textsc{SYS} is the system's estimate, and \textsc{LS} the post processed least squares estimate. The label 7 denotes the end of the experiment after rotating the phone in multiple directions.

In all our test cases the system calibration was not able to calibrate the hard iron effect while walking on the test track. Only in the last phase of the experiment when the device was rotated in multiple directions. The estimates of the systems were very close to the least squares estimates.

Figure \ref{fig:eval_simulation_pixel3} and Table \ref{tbl:eval_simulation_pixel3} show the convergence of the hard iron estimation for the Google Pixel 3. The particle filter calibration performs well but picks up a bias on the z axis after label 4 and during the calibration phase at the end there is a bias on the x axis.

\begin{figure}[H]
    \centering
    \includegraphics[width=1.0\textwidth]{figures/convergence_pixel3.png}
    \caption{Estimated hard iron bias and error over time on the Google Pixel 3.}
    \label{fig:eval_simulation_pixel3}
\end{figure}

\begin{table}[H]
    \centering
    \resizebox{\columnwidth}{!}{
    \begin{tabular}{ | l | l | r | r | r | r | r | r | r | }
    \hline
    \textbf{Method} & \textbf{Axis} & \textbf{1} & \textbf{2} & \textbf{3} & \textbf{4} & \textbf{5} & \textbf{6} & \textbf{7} \\ \hline
    PF  & X & $-0.5\pm100.5$ & $-9.3\pm24.0$ & $-5.8\pm7.2$ & $-5.3\pm4.6$ & $-5.3\pm3.9$ & $-4.9\pm3.7$ & $-8.4\pm1.1$ \\ \hline
    SYS &   & $1.0$ & $1.0$ & $1.0$ & $1.0$ & $1.0$ & $1.0$ & $-6.4$ \\ \hline
    LS  &   & & & & & & & $-5.4$ \\ \hline
    PF  & Y & $-0.6\pm100.4$ & $-3.7\pm21.6$ & $-3.3\pm8.8$ & $-2.4\pm5.0$ & $-2.3\pm4.1$ & $-2.4\pm4.0$ & $-0.9\pm1.1$ \\ \hline
    SYS &   & $32.6$ & $32.6$ & $32.6$ & $32.6$ & $32.6$ & $32.6$ & $1.3$ \\ \hline
    LS  &   & & & & & & & $0.3$ \\ \hline
    PF  & Z & $1.2\pm100.8$ & $52.1\pm22.5$ & $53.4\pm15.4$ & $61.5\pm8.1$ & $63.3\pm5.8$ & $63.7\pm5.4$ & $58.9\pm1.1$ \\ \hline
    SYS &   & $8.4$ & $8.4$ & $8.4$ & $8.4$ & $8.4$ & $8.4$ & $59.4$ \\ \hline
    LS  &   & & & & & & & $60.1$ \\ \hline
    \end{tabular}
    }
    \caption{Estimated hard iron bias in $\mu T$ with different methods on the Google Pixel 3.}
    \label{tbl:eval_simulation_pixel3}
\end{table}

Figure \ref{fig:eval_simulation_pixel2} and Table \ref{tbl:eval_simulation_pixel2} show the convergence of the hard iron estimation for the Google Pixel 2. The particle filter calibration looks unbiased until label 6 but picks up a bias during the calibration phase at the end. There we get a small bias for the x and y axis.

\begin{figure}[H]
    \centering
    \includegraphics[width=1.0\textwidth]{figures/convergence_pixel2.png}
    \caption{Estimated hard iron bias and error over time on the Google Pixel 2.}
    \label{fig:eval_simulation_pixel2}
\end{figure}

\begin{table}[H]
    \centering
    \resizebox{\columnwidth}{!}{
    \begin{tabular}{ | l | l | r | r | r | r | r | r | r | }
    \hline
    \textbf{Method} & \textbf{Axis} & \textbf{1} & \textbf{2} & \textbf{3} & \textbf{4} & \textbf{5} & \textbf{6} & \textbf{7} \\ \hline
    PF  & X & $1.4\pm99.7$ & $62.3\pm9.1$ & $63.4\pm6.7$ & $62.7\pm5.1$ & $62.6\pm4.7$ & $59.8\pm1.7$ & $58.7\pm1.0$ \\ \hline
    SYS &   & $63.6$ & $63.6$ & $63.6$ & $63.6$ & $63.6$ & $63.6$ & $62.0$ \\ \hline
    LS  &   & & & & & & & $63.3$ \\ \hline
    PF  & Y & $0.9\pm100.1$ & $-68.7\pm11.4$ & $-70.0\pm7.6$ & $-69.1\pm6.2$ & $-68.8\pm5.8$ & $-64.2\pm1.9$ & $-67.9\pm1.0$ \\ \hline
    SYS &   & $-45.4$ & $-45.4$ & $-45.4$ & $-45.4$ & $-45.4$ & $-45.4$ & $-71.5$ \\ \hline
    LS  &   & & & & & & & $-71.2$ \\ \hline
    PF  & Z & $1.1\pm99.9$ & $1.4\pm17.7$ & $-1.2\pm11.7$ & $-0.5\pm9.2$ & $-0.0\pm8.2$ & $-3.1\pm1.6$ & $-0.2\pm1.0$ \\ \hline
    SYS &   & $-27.4$ & $-27.4$ & $-27.4$ & $-27.4$ & $-27.4$ & $-27.4$ & $-0.6$ \\ \hline
    LS  &   & & & & & & & $0.7$ \\ \hline
    \end{tabular}
    }
    \caption{Estimated hard iron bias in $\mu T$ with different methods on the Google Pixel 2.}
    \label{tbl:eval_simulation_pixel2}
\end{table}

Figure \ref{fig:eval_simulation_s7} and Table \ref{tbl:eval_simulation_s7} show the convergence of the hard iron estimation for the Samsung Galaxy S7. The hard iron effect of the z axis is about 150 $\mu$T and therefore quite challenging. The filter will not initialize a lot of particles in close proximity since the \textsc{initialVariance} is just 100 $\mu$T. The y axis picks up a similar bias most likely due to the same reason as the z axis. During calibration the accuracy increases but a small bias remains on the x and y axis.

\begin{figure}[H]
    \centering
    \includegraphics[width=1.0\textwidth]{figures/convergence_s7.png}
    \caption{Estimated hard iron bias and error over time on the Samsung Galaxy S7.}
    \label{fig:eval_simulation_s7}
\end{figure}

\begin{table}[H]
    \centering
    \resizebox{\columnwidth}{!}{
    \begin{tabular}{ | l | l | r | r | r | r | r | r | r | }
    \hline
    \textbf{Method} & \textbf{Axis} & \textbf{1} & \textbf{2} & \textbf{3} & \textbf{4} & \textbf{5} & \textbf{6} & \textbf{7} \\ \hline
    PF  & X & $1.1\pm100.0$ & $-20.9\pm17.2$ & $-24.1\pm7.0$ & $-24.9\pm5.5$ & $-25.8\pm4.4$ & $-26.1\pm4.2$ & $-27.2\pm1.1$ \\ \hline
    SYS &   & $0.0$ & $0.0$ & $0.0$ & $0.0$ & $0.0$ & $0.0$ & $-25.6$ \\ \hline
    LS  &   & & & & & & & $-26.1$ \\ \hline
    PF  & Y & $1.8\pm100.2$ & $-27.0\pm19.8$ & $-32.4\pm7.6$ & $-31.9\pm4.5$ & $-32.0\pm3.3$ & $-32.1\pm3.2$ & $-20.3\pm1.1$ \\ \hline
    SYS &   & $0.0$ & $0.0$ & $0.0$ & $0.0$ & $0.0$ & $0.0$ & $-23.6$ \\ \hline
    LS  &   & & & & & & & $-21.9$ \\ \hline
    PF  & Z & $0.9\pm99.9$ & $-163.8\pm16.7$ & $-167.7\pm8.9$ & $-169.8\pm6.4$ & $-170.7\pm5.1$ & $-171.1\pm4.9$ & $-154.9\pm1.2$ \\ \hline
    SYS &   & $0.0$ & $0.0$ & $0.0$ & $0.0$ & $0.0$ & $0.0$ & $-154.8$ \\ \hline
    LS  &   & & & & & & & $-154.6$ \\ \hline
    \end{tabular}
    }
    \caption{Estimated hard iron bias in $\mu T$ with different methods on the Samsung Galaxy S7.}
    \label{tbl:eval_simulation_s7}
\end{table}

Figure \ref{fig:eval_simulation_nexus5x} and Table \ref{tbl:eval_simulation_nexus5x} show the convergence of the hard iron estimation for the LG Nexus 5X. Here the calibrations performs reasonable well and remains in $\pm \sigma$ until calibration phase. The estimation seems to pick up oscillations from the rotations of the devices. This might be due to a bias in the orientation estimation. \textsc{beta} might not be chosen optimal for this device.

\begin{figure}[H]
    \centering
    \includegraphics[width=1.0\textwidth]{figures/convergence_nexus5x.png}
    \caption{Estimated hard iron bias and error over time on the LG Nexus 5X.}
    \label{fig:eval_simulation_nexus5x}
\end{figure}

\begin{table}[H]
    \centering
    \resizebox{\columnwidth}{!}{
    \begin{tabular}{ | l | l | r | r | r | r | r | r | r | }
    \hline
    \textbf{Method} & \textbf{Axis} & \textbf{1} & \textbf{2} & \textbf{3} & \textbf{4} & \textbf{5} & \textbf{6} & \textbf{7} \\ \hline
    PF  & X & $0.4\pm99.4$ & $9.9\pm21.1$ & $7.1\pm9.6$ & $11.6\pm6.1$ & $11.3\pm4.4$ & $10.4\pm4.2$ & $2.9\pm1.1$ \\ \hline
    SYS &   & $-4.4$ & $-4.4$ & $-4.4$ & $-4.4$ & $-4.4$ & $-4.4$ & $10.6$ \\ \hline
    LS  &   & & & & & & & $9.2$ \\ \hline
    PF  & Y & $2.0\pm99.8$ & $-8.8\pm20.9$ & $-5.9\pm10.6$ & $-6.4\pm6.6$ & $-6.7\pm5.1$ & $-6.3\pm4.6$ & $-2.6\pm1.2$ \\ \hline
    SYS &   & $-47.8$ & $-47.8$ & $-47.8$ & $-47.8$ & $-47.8$ & $-47.8$ & $-6.7$ \\ \hline
    LS  &   & & & & & & & $-5.1$ \\ \hline
    PF  & Z & $-1.5\pm100.1$ & $45.9\pm20.7$ & $44.8\pm14.3$ & $40.6\pm10.3$ & $39.2\pm8.2$ & $39.8\pm7.7$ & $48.1\pm1.4$ \\ \hline
    SYS &   & $42.8$ & $42.8$ & $42.8$ & $42.8$ & $42.8$ & $42.8$ & $47.2$ \\ \hline
    LS  &   & & & & & & & $47.0$ \\ \hline
    \end{tabular}
    }
    \caption{Estimated hard iron bias in $\mu T$ with different methods on the LG Nexus 5X.}
    \label{tbl:eval_simulation_nexus5x}
\end{table}

In all our test cases the particle filter was able to converge. The estimated error dropped to approximately $\pm$ 6 $\mu$T after 20 seconds while producing biases of about $\pm$ 10 $\mu$T in worst case. The accuracy of our filter was worse compared to the system and least squares but it produces results when the other methods did not. The biases might come from a previous signal processing step like the moving average or the orientation filter. Compared to the earth magnetic field the errors are reasonable small and should not effect the compass too dramatically.

\subsection{Performance}

Since we are targeting mobile devices and want to run our particle filter in real time we have to evaluate its performance. A faster algorithm has the benefit of being compatible with older devices and will consume less battery. Depending on the use case the performance can be tweaked with the amount of particles. Reducing the population of the particle filter will also reduce its stability. For a practical use case it is important to find a sweet spot with optimal performance and stability.

The parameter \textsc{minimalRotation} was set to zero to get an upper limit for the CPU consumption. Otherwise the parameters were the same as shown in Table \ref{tbl:eval_params}.

The performance measurements are summarized in Table \ref{tbl:eval_performance}. Each value is the result of a 10 second measurement of the CPU time spent in the particle filter divided by real-time passed.

\begin{table}[h]
    \centering
    \begin{tabular}{ | l | r | r | r | r | r | }
    \hline
    \textbf{Device \textbackslash \ Population} & $10^3$ & $10^4$ & $10^5$ \\ \hline
    Google Pixel 3    & 0.00 & 0.06 & 0.51 \\ \hline
    Google Pixel 2    & 0.01 & 0.11 & 0.53  \\ \hline
    Samsung Galaxy S7 & 0.01 & 0.14 & 0.79  \\ \hline
    LG Nexus 5X       & 0.01 & 0.10 & 0.59  \\ \hline
    \end{tabular}
    \caption{CPU usage per device and particle filter population.}
    \label{tbl:eval_performance}
\end{table}
