The goal of this thesis was to overcome limitations of the \gls{os} hard iron calibration of smartphones in pedestrian navigation, localization and wayfinding scenarios, like \gls{ips} (more details in Section \ref{sec:ips}). We identified the following major limitations: different algorithms across different devices, dependence on gestures that have to be communicated to and performed by the user, no continuous calibration, and no quantified error estimation.

We proposed a particle filter to calibrate the hard iron effect of three-axis magnetometers in smartphones by sensor fusion with the accelerometer and gyroscope. Our algorithm can be updated iteratively and takes previous estimations into account. It does not depend on specific gestures that have to be communicated to the user. Any kind of rotations of the device are taken into account. Additionally, we can obtain a quantified error estimate for the hard iron calibration which can be used for error propagation in other filters that depend on the magnetic field (e.g. a compass).

A test plan was developed with a predefined track and different calibration phases to make the measurements of the different test devices comparable. We compared the results of the particle filter to those of the \gls{os} and to least squares estimates for reference. 

Our particle filter is producing promising results with a reasonable amount of CPU usage. It was able to calibrate the hard iron effect just by walking on the test track while the \gls{os} was not able to produce any estimates. This promotes our algorithm to a viable solution for online calibration of magnetometers in smartphones in navigation, localization and wayfinding scenarios. There is still a lot of room for improvements which are discussed in Chapter \ref{ch:outlook}.

As discovered during the work on this thesis, not all Android devices with a magnetometer will offer uncalibrated readings. Unfortunately, this limits the scope of application for our particle filter. Possible solutions are discussed in Chapter \ref{ch:outlook}.
