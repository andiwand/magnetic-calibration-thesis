Nowadays smartphones usually come with a built-in magnetometer that can be used to estimate the horizontal orientation of the phone like a compass. Since phones contain magnetic parts like speakers it will source its own magnetic field that will bias the measurement of the magnetometer. A calibration is necessary to estimate the bias and to subtract it from the measurement. Such a calibration usually requires sensor fusion with accelerometer and gyroscope input and is implemented by the operating system or manufacturer. These implementations are optimized for low computational effort and fast manual rotations in all directions of the phone by the user. Since there are a lot of different phones on the market the implementations will behave differently and in general it will not be clear when the calibration is finished.

The goal of this thesis is to use a particle filter to realize a different type of magnetic calibration that is performed continuously and without forcing the user to execute special gestures. In contrast to standard calibration techniques, this calibration also provides an estimate of the calibration quality over time. If the computational effort is too big further constraints will be used like known magnetic field amplitude at the starting position.

Currently, we use the magnetic flux density vector to estimate the horizontal orientation of the user directly and ignore the systematic errors. The orientation is displayed as an arrow on the map and will confuse the user if it points in the wrong direction. Furthermore, the error will propagate through the position estimation in a later stage.
