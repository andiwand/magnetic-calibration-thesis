% TODO maybe use figures and examples across the chapter

This chapter is a short introduction to \gls{smc} methods and specifically particle filters. First, we give an overview about particle filters and state-space models. Then we will derive the recursive filtering equations by following Bayes' theorem. At the end, we will introduce resampling and point out why it is a crucial feature of \gls{smc} methods.

\section{Overview}

The term \textit{filtering} in this context means extracting information about a signal from partial and noisy observations in dynamical systems. In contrast to other estimation problems, filtering is about estimating the current state of a system given only observations up to this point in time.\cite{AppliedOptimalEstimation}

Particle filters were originally developed for object tracking and time series analysis for nonlinear, non-Gaussian state-space models.\cite{Gordon1993} The term particle filter was first used by Del Moral in 1996.\cite{Moral1996}

Particle filters are \gls{smc} methods and a subset of \gls{mc} algorithms. They can be used to solve filtering problems arising in signal processing and Bayesian statistical inference. The goal is to compute the posterior distributions of states in a Markov process, given the prior distribution of states and some partial and noisy observations.\cite{Doucet2011}

The particle filter uses a set of particles to represent the current state of the system. There is no limitation to the state-space model, the initial state, and the noise distribution of observations. However, in practice the particle filter does not perform well in very high-dimensional systems.\cite{Doucet2011}

Each particle has a weight (or likelihood) that represents an approximate probability measure of that particle being sampled from the usually analytically unknown probability density function. The weights are updated when new evidence about the true state is available. A resampling step is necessary to avoid weight disparity leading to weight collapse. There are several adaptive resampling criteria that can be used to reduce noise, including the variance of the weights and the relative entropy with respect to the uniform distribution. In the resampling step, the particles with negligible weights are replaced by new particles in the proximity of the particles with higher weights.\cite{Moral1996}\cite{Moral2012}

\section{Filtering with state-space models}

According to Hans Künsch\cite{Kuensch2013}, in the last 50 years filtering has been mainly studied in the framework of state-space or hidden Markov models, assuming a Markovian time evolution of the signal and observations, which are instantaneous functions of the signal, and subject to white observation noise.

The objective of a filter is to estimate the hidden state variables $X_k$ (also called signal) given the observation variables $Y_k$ at iteration $k$. The observable variables $Y_k$ are related to the hidden variables $X_k$ by some conditional probability density $g(y_k|x_k)$ that is known. Similarly, the dynamics of the state process is also known $f(x_k|x_{k-1})$. A generic state and observation process is illustrated below.\cite{Kuensch2013}

\[\begin{array}{cccccccccc}
X_0 & \to & X_1 & \to & X_2 & \to & X_3 & \to & \cdots & \text{signal} \\
\downarrow & & \downarrow & & \downarrow & & \downarrow & & \cdots & \\
Y_0 & & Y_1 & & Y_2 & & Y_3 & & \cdots & \text{observation}
\end{array}\]

Given the sequential observations $Y_0,\cdots,Y_k$ at any time step $k$, the filtering problem is to estimate the signal $X_k$.

Bayesian estimates of $X_k$ follow from the posterior density $p(x_k|y_0,\cdots,y_k)$. Particle filters provide an approximation of these conditional probabilities by weighted samples. In contrast to \gls{smc} methods, \gls{mcmc} methods would have to deal with the full posterior distribution $p(x_0,\cdots,x_k|y_0,\cdots,y_k)$ in each iteration.

\section{Filtering recursion}

In order to start the recursion, a prior distribution $p(x_0)$ of the signal $X_0$ is required. Selecting samples of the prior distribution is called initialization. One has to be careful choosing a prior distribution because it might cause bias. The prior could depend on the first observation $Y_0$ for example.

At time step $k=1$ we observe $Y_1$ and want to update the filter. Following Bayes' theorem (see Equation \ref{eq:bayes} in Section \ref{sec:bayes}) we derive the following:

\begin{equation}
\label{eq:pf_1}
    p(x_0,x_1|y_1) \sim L(y_1|x_0,x_1)\ p(x_1) = g(y_1|x_1)\ f(x_1|x_0)\ p(x_0)
\end{equation}

Where $L(y_1|x_0,x_1)$ is the likelihood function given by the observation function $g(y_k|x_k)$ and $p(x_1)$ is the prior for $X_1$ given by the signal prediction function $f(x_k|x_{k-1})$ and $p(x_0)$. Following this process recursively and applying normalization yields the following:

\begin{equation}
\label{eq:pf}
    p(x_{1:k}|y_{1:k}) = \frac{g(y_k|x_k)\ f(x_k|x_{k-1})\ p(x_{1:k-1}|y_{1:k-1})}{p(y_k|y_{1:k-1})}
\end{equation}

\begin{equation}
\label{eq:pf_norm}
    p(y_k|y_{1:k-1}) = \int_{S} g(y_k|x_k)\ f(x_k|x_{k-1})\ p(x_{1:k-1}|y_{1:k-1}) \,dx_k
\end{equation}

With the abbreviation $x_{1:k} = x_1,\cdots,x_k$, the normalization $p(y_k|y_{1:k-1})$ in Equation \ref{eq:pf_norm}, and the state space volume $S$.

By marginalization of \ref{eq:pf} (integration over $x_{1:k-1}$) we derive the following:

\begin{equation}
\label{eq:pf_update}
    p(x_k|y_{1:k}) = \frac{g(y_k|x_k)\ p(x_k|y_{1:k-1})}{p(y_k|y_{1:k-1})}\
\end{equation}

\begin{equation}
\label{eq:pf_prediction}
    p(x_k|y_{1:k-1}) = \int_{V} f(x_k|x_{k-1})\ p(x_{k-1}|y_{1:k-1}) \,dx_{k-1}
\end{equation}

Equation \ref{eq:pf_prediction} is usually called the prediction step and \ref{eq:pf_update} the update step. However, according to Arnaud Doucet\cite{Doucet2011} most particle filtering methods rely on numerical approximations of \ref{eq:pf} instead of \ref{eq:pf_prediction} and \ref{eq:pf_update}.

Analytical solutions can be given for two important special cases. First, when the state space $S$ is finite, the integrals above can be reduced to finite sums. Second, when the system is linear and $p(x_0)$, $g(y_k|x_k)$, and $f(x_k|x_{k-1})$ are Gaussian, the filtering problem can be solved by a Kalman filter exactly.\cite{Kalman1960}

\section{Resampling}
\label{sec:pf_resampling}

An algorithm that follows the initialization, prediction, and update steps sequentially by representing the distributions with $N$ samples is called \gls{sis}. It can be shown that the variance of the weights is growing exponentially with the number of iterations. The important and defining feature of particle filters is the resampling step, which removes particles with low weights and replaces particles with high weights by multiple offspring particles in close proximity. After a resampling step, all particles have the same weight and the distribution is only represented by the number of particles in the different states.\cite{Doucet2011}

Common algorithms for resampling are \textbf{systematic resampling} and \textbf{multinomial resampling}. Both require a summation of all weights which can be a bottleneck for parallelization. Other techniques were discussed by Murray, Lee, and Jacob in 2016.\cite{Doucet2011}\cite{parallel_resampling}

Adaptive resampling can be used to reduce noise and avoid unnecessary resampling steps in case of a sufficiently even weight distribution. A popular criteria is the effective number of particles.

\begin{equation}
\label{eq:pf_effective}
    {\hat{N}}_{\mathit{eff}} = \frac{1}{\sum_{i=1}^N\left(w_i\right)^2} \leq N_\mathit{thr}
\end{equation}

With normalized weights \(\sum_{i=1}^N{w_i} = 1\) and the threshold parameter $N_\mathit{thr}$.\cite{Moral2012}
